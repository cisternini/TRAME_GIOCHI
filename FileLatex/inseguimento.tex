
\paragraph{Inseguimento}
\subparagraph*{Iniziare inseguimento}

\begin{itemize}
    \item \textbf{Condizioni per un inseguimento}: Necessita di una preda e almeno un inseguitore.
    \item \textbf{Iniziativa}: I partecipanti non già nell’ordine di iniziativa devono tirare per l’iniziativa.
    \item \textbf{Turni}: Ogni partecipante può effettuare un’azione e muoversi nel proprio turno.
    \item \textbf{Termine dell'inseguimento}: Finisce quando una delle parti abbandona l'inseguimento o la preda riesce a fuggire.
    \item \textbf{Distanza iniziale}: Il DM determina la distanza iniziale tra preda e inseguitori.
    \item \textbf{Controllo della distanza}: Il DM tiene traccia della distanza durante l'inseguimento.
    \item \textbf{Primo inseguitore}: L'inseguitore più vicino alla preda viene designato come primo inseguitore, che può cambiare ogni round.
\end{itemize}

\subparagraph*{Condurre inseguimento}

\begin{itemize}
    \item \textbf{Motivazione all'azione di Scatto}: I partecipanti all'inseguimento sono incentivati a usare l'azione di Scatto in ogni round.
    \item \textbf{Rischi di fermarsi}: Inseguitori che si fermano per attaccare o lanciare incantesimi possono perdere la preda, e viceversa.
    \item \textbf{Azione di Scatto}: 
    \begin{itemize}
        \item Ogni partecipante può utilizzare l'azione di Scatto fino a 3 + modificatore di Costituzione.
        \item Ogni azione di Scatto aggiuntiva richiede una prova di Costituzione CD 10; in caso di fallimento, si subisce un livello di indebolimento.
        \item Un partecipante abbandona l'inseguimento dopo 5 livelli di indebolimento (velocità ridotta a 0).
        \item I livelli di indebolimento possono essere rimossi con un riposo breve o lungo.
    \end{itemize}
\end{itemize}

\subparagraph*{Attacchi e incantesimi}

\begin{itemize}
    \item \textbf{Attacchi e incantesimi}: Possono essere effettuati attacchi e lanci di incantesimi contro le creature a portata.
    \item Si applicano le normali regole relative a copertura e terreno.
    \item \textbf{Attacchi di opportunità}: 
    \begin{itemize}
        \item Normalmente, non possono esserci attacchi di opportunità tra i partecipanti all'inseguimento, poiché si muovono nella stessa direzione.
        \item Possono subire attacchi di opportunità da creature non coinvolte nell'inseguimento.
        \item Esempio: Un avventuriero che insegue un furfante potrebbe subire attacchi di opportunità da parte di malviventi presenti nel vicolo.
    \end{itemize}
\end{itemize}

\subparagraph*{Termine dell'inseguimento}

\begin{itemize}
    \item \textbf{Un inseguimento termina quando}: 
    \begin{itemize}
        \item Una delle parti coinvolte si ferma.
        \item La preda fugge.
        \item Gli inseguitori si avvicinano abbastanza da prendere la preda.
    \end{itemize}
    \item \textbf{Prova di Furtività}: 
    \begin{itemize}
        \item La preda effettua una prova di Destrezza (Furtività) alla fine di ogni round, dopo che tutti i partecipanti hanno agito.
        \item Il risultato è confrontato con i punteggi di Saggezza (Percezione) passiva degli inseguitori.
        \item Se ci sono più prede, tutte devono effettuare una prova.
    \end{itemize}
    \item \textbf{Condizioni di successo}: 
    \begin{itemize}
        \item Se il primo inseguitore tiene sempre d'occhio la preda, la prova fallisce automaticamente.
        \item Se il risultato della prova di Furtività è superiore al punteggio passivo più alto degli inseguitori, la preda fugge.
        \item Altrimenti, l'inseguimento continua per un altro round.
    \end{itemize}
    \item \textbf{Vantaggio e svantaggio}: 
    \begin{itemize}
        \item La preda può ottenere vantaggio o svantaggio alla prova a seconda delle circostanze (consultare la tabella "Fattori di Fuga").
        \item Se la preda ha sia vantaggio che svantaggio, non ottiene né l'uno né l'altro.
    \end{itemize}
\end{itemize}

\subparagraph*{Complicazioni negli inseguimenti}

\begin{itemize}
    \item \textbf{Complicazioni negli inseguimenti}: Possono rendere l'inseguimento più frenetico e imprevedibile.
    \item \textbf{Tabelle di complicazioni}: 
    \begin{itemize}
        \item Consultare la tabella "Complicazioni degli Inseguimenti Urbani".
        \item Consultare la tabella "Complicazioni degli Inseguimenti nelle Terre Selvagge".
    \end{itemize}
    \item \textbf{Determinazione delle complicazioni}: 
    \begin{itemize}
        \item Ogni partecipante tira un d20 alla fine del proprio turno.
        \item Se si verifica una complicazione, questa influisce sul successivo partecipante nell'ordine di iniziativa.
    \end{itemize}
    \item \textbf{Negare la complicazione}: 
    \begin{itemize}
        \item Il partecipante che tira il dado o il successivo possono usare ispirazione per negare la complicazione.
    \end{itemize}
    \item \textbf{Complicazioni personali}: I personaggi possono creare complicazioni personali per allontanarsi dagli inseguitori (es. usando l'incantesimo ragnatela in un vicolo stretto).
    \item \textbf{Valutazione delle complicazioni}: Il DM decide come gestire le complicazioni create dai personaggi.
\end{itemize}

\begin{longtable}{|c|p{12cm}|}
    \hline
    \textbf{D20} & \textbf{Complicazione} \\
    \hline
    1 & Un albero caduto blocca il passaggio. Occorre fare un’azione per scavalcarlo con una prova di Atletica CD 15 o trovare un percorso alternativo. \\
    2 & Un gruppo di rovi spinosi rallenta il movimento. Il terreno è difficile per 6 metri e infligge 1d4 danni da perforazione. \\
    3 & Un tronco scivoloso costringe il personaggio a effettuare una prova di Acrobazia CD 14 per non cadere prone. \\
    4 & Un branco di animali selvatici attraversa il percorso, costringendo a un tiro salvezza su Destrezza CD 13 per evitarli. \\
    5 & Una buca nascosta richiede un tiro salvezza su Destrezza CD 12 per non inciampare e cadere prone. \\
    6 & Un tratto di fango rallenta il movimento. Il terreno diventa difficile per 9 metri. \\
    7 & Radici sporgenti possono far inciampare. Prova di Acrobazia CD 12 per evitarle. \\
    8 & Un albero a metà caduta oscilla pericolosamente. Prova di Percezione CD 10 per accorgersene e evitarlo. \\
    9 & Una fitta nebbia riduce la visibilità a 3 metri per il prossimo turno. \\
    10 & Un improvviso suono nella foresta distrae i personaggi. Prova di Saggezza (Percezione) CD 10 per non perdere tempo altrimenti subisce svantaggio al prossimo tiro. \\
    \hline
\end{longtable}


\begin{itemize}
    \item \textbf{Separazione delle creature}: Le creature inseguite possono dividersi in gruppi più piccoli per aumentare le possibilità di fuga.
    \item \textbf{Costringe gli inseguitori}: Questa tattica costringe gli inseguitori a decidere se dividere le loro forze o permettere a alcune prede di fuggire.
    \item \textbf{Gestione degli inseguimenti}:
    \begin{itemize}
        \item Se l'inseguimento si divide in più inseguimenti, ognuno si risolve separatamente.
        \item Il DM gestisce un round per ciascun inseguimento, seguendo l'ordine e tenendo conto delle distanze per ogni gruppo.
    \end{itemize}
\end{itemize}



