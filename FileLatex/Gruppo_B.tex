\documentclass{article}
\usepackage{graphicx} % Required for inserting images
\usepackage{wrapfig}
\usepackage{hyperref}
\usepackage[export]{adjustbox}
\usepackage{geometry}
\setcounter{secnumdepth}{5}
\setcounter{tocdepth}{5}
\geometry{a4paper,
 total={170mm,257mm},
 left=5mm,
 top=20mm}

\title{Gruppo B (Federico, Antonio, Pier, Angelo)}
\author{Giovanni Cisternini}
\date{December 2023}

\begin{document}
\tableofcontents

\maketitle


\section{Trama}
  









\newpage
        \subsection{La Torre del Mago (da Liv 1 a liv 2)}
            \subsubsection{Sotterranei}
Scendendo, i personaggi arrivano in un corridoio costruito in pietra. Alcune torce emanano una luce tenue. Coloro che hanno Scurovisione riescono a vedere come se il luogo fosse illuminato. Gli altri hanno difficoltà a vedere come il corridoio prosegua. Il DM chiede ai giocatori di effettuare una prova di Saggezza (Percezione) con una Classe Difficoltà (CD) di 10. I personaggi senza Scurovisione hanno svantaggio. Con un successo sulla prova, i personaggi vedranno che, sul muro, dopo 9 metri, c'è una leva al posto di una torcia.Quando viene tirata la leva, si apre un passaggio segreto nel muro. Il passaggio conduce al dungeon che si trova nei sotterranei della torre (mostrato nella pianta qui a fianco). I numeri nella piatina corrispondono alla seguente lista dei luoghi. I quadrati nella mappa hanno un lato di 1,5 metri.
\begin{enumerate}
    \item \textbf{Ingresso} Entrando in questa stanza, i personaggi sentiranno il passaggio segreto che si richiude alle loro spalle. Sono prigionieri nel dungeon. Una voce echeggia nella stanza: \textit{Benvenuti nella mia torre amici. Sono così felice di avere finalmente degli ospiti per provare la mia splendida creazione.
Mi domando solo quanto a lungo sopravviverete.} \textsc{Questa prima stanza è grande e vuota. Il pavimento è bagnato e potete sentire l'eco del rumore di gocce d'acqua che cadono. Alcune torce illuminano questo luogo. Potete distinguere due porte sulla parete nord.}
    \item \textbf{Fossa} Anche questa stanza è vuota ma al centro c'è una trappola:
        \begin{itemize}
            \item Tiro su \textsc{Indagare}per scoprire la trappola CD 12
            \item TS Destrezza se ci cade CD 10
        \end{itemize}
    \item \textbf{Corridoio} \textsc{Entrando nel corridoio vedete quattro statue lungo i muri. Tra le statue ci sono delle torce appese e alcune porte. Potete sentire un movimento d'aria che suggerisce la presenza di un'apertura nella direzione della fine del corridoio.}
    \item \textbf{Trappola Velenosa} Aprire la porta equivale a far scattare la trappola (un ago sparato dal muro che attraversa la stanza e colpisce chi ha aperto la porta
        \begin{itemize}
            \item Tiro per colpire Trappola : d20 + 4
            \item TS Costituzione per non essere Avvelenato(svantaggio ai tiri per colpire e alle prove abilita finché non beve l'antidoto) CD 10
            \item \textsc{Indagare} sulla porta CD 15 trova la trappola
            \item \textsc{Rapidita di Mano} per disinnescare la trappola CD 18
            \item se entrando trovano uno scheletro con vestiti e borsa, nella borsa una pozione curativa (2d4+2)
        \end{itemize}
    \item \textbf{Canile} Ci sono 2 cani rabbiosi appena entrano combattimento con Doberman \newline
         ci sono poi, sulla parete opposta alla porta 5 gabbie, della quale 2 vuote in alto, in basso negli angoli ci sono 2 gabbie dalla quale sporgono degli stivali (Penelope e PG Pier) 
        CD per Aprire le Gabbie 12, CA lucchetto 19 PF 2
       
    \item \textbf{Scale} \textit{Sono così felice che siate sopravvissuti al mio dungeon.
Adesso, potrete affrontare le altre prove che ho preparato per voi. Vediamo se riposerete al primo piano per l'eternità!}

\end{enumerate}
\newpage
            \subsubsection{Primo Piano}
    \textsc{Salendo le scale arrivate in una stanza ottagonale illuminata dalla grande vetrata che vedete di fronte a voi. I vetri colorati mostrano la figura di un drago.Sul pavimento c'è una grande cornice, delle stesse dimensioni della vetrata, e vari pezzi di legno di diverse forme.}
            \subsubsection{Secondo Piano}
\textsc{Due grosse lucertole, in piedi sugli arti posteriori, vi aspettano.
Una è robusta e ben equipaggiata, sta in mezzo alla stanza.
L'altra è più esile e mantiene la distanza. "Benvenuti al piano
della morte" sibila il secondo lucertoloide mentre incocca una
freccia nel suo arco.}

\begin{enumerate}
    \item Shkiss ha una pozione antidoto
    \item Sgrolt ha una pozione curativa (2d4+2) 
\end{enumerate}

            \subsubsection{Terzo Piano}
un'Altra stanza ottagonale... una stanza molto disordinata piena di scatole e barili. Nella parete opposta all'entrata c'è una porta. quando si avvicinano, si accorgono che è chiusa da 4 serrature. 
\begin{enumerate}
    \item Percezione CD 12, Una chiave si trova appesa al soffitto, ad una altezza di 3 metri. Si puo raggiungere scalando la parete con Atletica CD 15 (se cade 1d6 di danni), Se provano a tagliare la corda lanciando qualcosa CD 20 1 PF
    (si abbassa se si avvicinano)
    \item 2 piccoli forzieri : 1 pozione curativa, 1 la chiava. Forzieri con 20 PF. CD 12 Rapidita di mano per aprire.
    \item Chiave in un blocco di ghiaccio, ci impiega 5 ore a sciogliersi, con una fiamma si scioglie in pochi minuti. 
    \item L'ultima chiave si trova in una scatola in legno con una facciata di vetro, da cui si vede la struttura di un labirinto e attaccata alla scatola c'è una catenella con un magnete, e sul fianco della scatola c'è un buco da cui far uscire la chiave. Distruggere la scatola con 20 PF.
    \end{enumerate}

    dietro la porta c`è la scala che porta all'ultimo piano.
            \subsubsection{Ultimo Piano}
\textsc{Miei cari ospiti, non mi aspettavo che sareste arrivati fino
all'ultimo piano della mia torre. Sono incantato di conoscervi
personalmente. Oggi è davvero un giorno speciale, ho avuto
l'opportunità di vedervi soffrire affrontando le mie prove e
avrò l'onore di porre fine alle vostre vite.}
Gli avventurieri
potrebbero provare a parlare col mago. Lui risponderà
ridendo e, se gli verranno chieste spiegazioni sul suo
comportamento, risponderà soltanto dicendo che si sta
divertendo.
Il mago non può lanciare incantesimi che danneggiano gli
avventurieri, ma userà la sua magia per impedire ai
personaggi di attaccare. Nel primo turno, usa Taumaturgia
nel tentativo di spaventare i personaggi.
Userà poi l'incantesimo Armatura Magica per aumentare la
propria CA oppure Immagine Speculare per schivare i colpi.
Successivamente lancia Interdizione alle Lame. Raggio di
Affaticamento previene i danni fisici, quindi il mago lo lancerà
contro Nadarr o Vidarr.
Risata Incontenibile di Tasha può essere usato per
bloccare Al . Charme su
Persone e Suggestione si rivelano utili per convincere a smettere di attaccare Si faccia attenzione alla durata degli incantesimi e alla
necessità di mantenere la concentrazione. Quando il mago
vuole danneggiare un personaggio, lo colpisce con lo scettro.
Il DM potrebbe evitare di farlo nei primi turni perché il mago
non ha molti punti ferita e, senza la protezione magica, non
riuscirebbe a sopravvivere a lungo.
Una volta sconfitto il mago, gli avventurieri sono
finalmente liberi. Nella stanza c'è una leva che apre il
passaggio segreto da cui i personaggi sono entrati nel
seminterrato. Troveranno inoltre 30 monete d'oro, gemme
per un valore di 20 monete d'oro e il libro degli incantesimi
del mago che contiene gli incantesimi usati dal mago stesso.

        \subsection{il Re rospo} Nadar e Vidar vengono mandati vicino Neverwinter da Thorgrim, perché gli è arrivata una richiesta di aiuto da un suo vecchio conoscente (Fili-hip). Partono per un villaggio anonimo all'interno della foresta di Neverwinter. La richiesta è vaga, si parla di uomini rana , re rospi e rapimenti vari.  arrivano dal capo villaggio di nome Fili-hip, che spiega : gli uomini-rana hanno sempre vissuto pacificamente nella palude, ma di recente si sono fatti arroganti e aggressivi e hanno cominciato ad assalire la comunità alla luce del giorno. Di solito queste creature si accontentano di rubare una gallina o due, ma da un paio di mesi a questa parte hanno preso ad attaccare i viaggiatori per derubarli. Nelle ultime due settimane si sono spunti a rapite dei contadini: quattro persone sono già state trascinate nella palude!  i termini che impone sono troppo onerosi per i contadini: vuole 10 pezzi d’oro oppure 2 armi di metallo in buone condizioni per ogni persona che verrà riportata alla famiglia. Fornisce anche le indicazioni per raggiungere la palude. 
            \subsubsection{Palude} Raggiunta la palude tiro su Sopravvivenza/Indagare CD 15 per trovare/notare le tracce di un uomo-rana e delle salamandre; se falliscono se lo trovano difronte dopo 200 metri, altrimenti riescono a seguirlo e tiro Furtivita per attacco a sorpresa. 
\begin{itemize}
    \item 1 \hyperlink{uomorana}{uomo rana}
    \item 2 salamandre giganti
    \item 4 \hyperlink{granchio}{granchi giganti}
\end{itemize}




Dopo questi scontri i personaggi: possono interrogare l’eventuale uomo-rana prigioniero per scoprire dove si trova la sua comunità, oppure possono cercare di seguire le tracce dei rapitori con una prova di Saggezza (Sopravvivenza) con Classe Difficoltà 12.
Se fallisce sbagliano strada e finiscono in uno stagno di alghe (Percezione CD 12 per accorgersene) dove trovano un banco di piccoli \hyperlink{sciame}{pesci carnivori}. Dopo aver ucciso i pesci, più in la vedono una casetta in paglia che poggia su un piccolo spiazzo di terra che sale sopra il livello della palude, se si avvicinano CD 14 Percezione passiva si accorgono di 2 paia di orme abbastanza grandi che escono dalla casa, se entrano e iniziano a rovistare :
\begin{itemize}
    \item In un bauletto aperto trovano 12 mr e una pozione guaritrice
    \item In un bauletto chiuso a chiave CD 13 Rapidita di mano, trovano 65 ma
\end{itemize}
\begin{itemize}
    \item se decidono di fare un riposo lungo chi deve salire al livello 2 e verso mezza notte si trovano 2 orchi che entrano in casa
    \item il più grosso ha addosso 12 mo e il più piccolo 18 di argento
\end{itemize}
Se non fallisce arrivano al villaggio degli uomini rana. Se decidono di riposare brevemente chi ha superato i 295-300 xp sale a livello 2 altrimenti usa il dado vita se vuole recuperare energie se fanno il riposo lungo tutti recuperano energia e chi ha superato gli XP sale a liv 2, ma durante la notte vengono attaccati da 2 orchi che si aggiravano nei dintorni.
La comunità degli uomini-rana è piuttosto selvaggia e spartana, non ci sono edifici veri e propri ma solo ripari di giunchi e qualche recinto dove vengono allevati granchi e gamberi di palude.
            \subsubsection{Villaggio}
Nel villaggio si trovano quattro uomini-rana combattenti: tre assaltano i personaggi e uno resta a protezione del \hyperlink{re}{Re dei Rospi}. Gli altri umanoidi anfibi non hanno il coraggio per combattere e sebbene numerosi non affronteranno i personaggi. Il Re è un rospo della taglia di un orso bruno e dotato di una mente acuta come quella di un umano, e del dono della parola.

                        \subparagraph{Comportamento}
I mostri più animaleschi si comportano in base all’istinto, attaccando la preda più facile e che offre minor resistenza. Gli uomini-rana invece utilizzano rudimentali tattiche per fiancheggiare i nemici, coglierli di sorpresa e tenerli separati. Se vedono un incantatore in azione cercano di eliminarlo per primo. Si tengono lontani da avversari pesantemente corazzati.


a
        \subsection{Waterdeep: il furto dei dragoni}

            \subsubsection{Volti Familiari}

                \paragraph{DURNAN}
N locandiere umano Illuskan

Il proprietario del Portale Spalancato è un avventuriero a riposo e un uomo di poche parole. Durnan (vedi l'appendice B) avverte bruscamente gli avventurieri di livello inferiore al 5° che entrare a Sottomonte “non è una buona idea”. Tiene sempre uno spadone magico nascosto sotto il bancone, nel caso qualcosa di mostruoso uscisse strisciando dal pozzo d'accesso.

                \paragraph{“BONNIE”}
N doppelganger camuffata da cameriera umana Tethyrian

Questa doppelganger che si spaccia per una cordiale cameriera è a capo di una banda di cinque doppelganger giunti a Waterdeep oltre un anno fa. Per aiutare il resto della banda ad arrotondare, ha deciso di lavorare per Durnan come cameriera. Mattrim Mereg (vedi sotto) conosce il suo segreto.

                \paragraph{MATTRIM “Tre CorpE” MEREG}
LB bardo umano Illuskan

Questo bardo poco socievole (vedi l'appendice B) si esibisce al Portale Spalancato ed è un musicista migliore di quanto lasci credere. La gente lo chiama “Tre Corde” perché suona un liuto a cui rimangono solo tre corde. In realtà, è una spia degli Arpisti ed è molto più eloquente e sofisticato di quanto lasci a vedere. Vive alla locanda e trascorre i pomeriggi e le serate a spiare gli agenti degli Zhentarim e a raccogliere informazioni su altri potenziali soggetti problematici. Di recente, ha fatto amicizia con Bonnie e vuole aiutare la sua banda di doppelganger a stabilirsi in città.

                \paragraph{JALESTER SILVERMANE}
LB guerriero umano Chondathan

Jalester Silvermane (vedi l'appendice B) è un agente dell'Alleanza dei Lord che fa rapporto direttamente a Laeral Silverhand. Il Lord Svelato di Waterdeep gli ha chiesto di spiare quegli avventurieri le cui azioni potrebbero aiutare o mettere in pericolo la città. Dal momento che il Portale Spalancato attira ogni genere di avventurieri, Jalester passa molto tempo al locale, solitamente standosene seduto da solo in un angolo. Durnan sa che Jalester lavora per Laeral e lascia stare il giovane. Jalester si lascia spesso distrarre dal pensiero di Faerrel Dunblade, il suo ragazzo, rimasto ucciso in una rissa di strada l’anno scorso. Quando non è in missione per conto dell'alleanza, Jalester si sente solo e va in cerca di un nuovo amore.

                \paragraph{MELOON WARDRAGON}
NM avventuriero umano Chondathan sotto il controllo di un divoracervelli

Meloon Wardragon (vedi l'appendice B) si presenta come un uomo allegro, ottimista e cordiale pronto a combattere al fianco dei suoi amici. Il personale e molti avventori regolari del Portale Spalancato sanno che Meloon è un abile guerriero legato alla Forza Grigia. Un divoracervelli in combutta con Xanathar ha divorato il cervello di Meloon alcuni mesi fa. Ora Meloon, sotto il controllo del mostro, si dà da fare per dissuadere gli avventurieri dall'esplorare Sottomonte e li incita a concentrarsi sui conflitti cittadini. Inoltre, dà la caccia agli agenti degli Zhentarim e li uccide per conto del suo padrone segreto, il beholder. Ha messo gli occhi su Davil Starsong (vedi l'appendice B), ma non ucciderà l’elfo sotto gli occhi di molti testimoni.

                \paragraph{OBAYA UDAY}
NB sacerdotessa di Waukeen umana Chultan

Obaya, una sacerdotessa, è giunta qui fin dal Chult per sovvenzionare una serie di spedizioni a Sottomonte con l’obiettivo di riportare più tesori magici possibile al suo datore di lavoro, il principe mercante Wakanga O’tamu di Porto Nyanzaru. Scoraggia gli avventurieri di basso livello dall’esplorare Sottomonte, ma è ben lieta di aiutarli con la sua magia finché non accumulano abbastanza esperienza da esserle utili. Se il DM intende condurre Waterdeep: Il Dungeon del Mago Folle dopo questa avventura, Obaya può svolgere un ruolo più prominente come consigliera e fonte di missioni.

                \paragraph{YAGRA STONEFIST}
N mezzorca malvivente prezzolata

Yagra è una mercenaria della Rete Nera ingaggiata per proteggere un negoziatore degli Zhent di nome Davil Starsong (vedi l'appendice B per ulteriori informazioni su di lui). Yagra trova il lavoro noioso e per passare il tempo sfida gli avventurieri a braccio di ferro (le gare si risolvono usando prove contese di Forza). Se i personaggi esprimono la loro opposizione alla Gilda di Xanathar, Yagra può incoraggiarli a parlare con Davil affinché uniscano le loro forze a quelle degli Zhentarim e distruggano il beholder signore del crimine. Yagra è una mezzorca malvivente. Quando ridotta a 0 punti ferita, scende invece a 1 punto ferita (ma non può farlo di nuovo finché non completa un riposo lungo). È dotata di scurovisione entro un raggio di 18 metri e parla il Comune e l'Orchesco.

            \subsubsection{Setting dell'avventura}
                        \subparagraph{}

                        \subparagraph{Nemico principale: } Manshoon (Inverno)

                        \subparagraph{Interdizione Draconica di Ahghairon}: I dragonidi hanno incubi durante la notte. 
\begin{itemize}
    \item \textbf{Al}: Si sveglia fra le montagne del Quibaluk, riconosce la sua stanza all'interno del monastero, si guarda le mani [battito di palpebre] vede le sue mani infuocate [battito di palpebre] si ritrova seduto con davanti a lui una tazza di te un vecchietto che gli da le spalle e dice "Figliolo cosa volevi dirmi dunque ..." il vecchietto si gira lentamente, e la faccia inizia a cadere e prendere fuoco. Il vecchio ha un ghigno sul viso e nel mentre inizia a ridere, delle voci sussurrano "Assassino, Mostro, Demone", [battitto di palpebre] tutto quanto inizia a prendere fuoco e le sue mani sono nuovamente ricoperte di fuoco, tutto intorno al lui prende fuoco, prova ad alzarsi ma c'è il cadavere del suo maestro con tutta la pelle carbonizzata che con voce rauca "Mi vendicherò demone" ... Si sveglia.
    \item \textbf{Nadarr}: Apre gli occhi e si ritrova al timone della barca di pirati, il cielo è sereno ed è tutto tranquillo, il suo amico Vidar è li al suo fianco che sorride e beve con una bottiglia di rum in mano. [chiude le palpebre] Si ritrova nel mare in tempesta, prova a navigare la nave, ma viene sbalzata a destra e sinistra, cerca Vidar per chiedere aiuto, ma nessuno risponde. Si accorge che sulla nave non c'è nessuno. Lascia il timone e inizia a correre per la nave, guarda in cabina... nessuno ... guarda in stiva ... nessuno ... si avvicina alla polena e trova il suo amico Vidar legato e pieno di sangue... urla forte appena tenta di avvicinarsi ... cade in acqua e inizia ad affogare... si sveglia. 
\end{itemize}


                        \subparagraph{Codice legale di Waterdeep}:  Pagina 223 del manuale il Furto dei dragoni.

            \subsubsection{Il Viaggio}
\textbf{Distanza N-W}: 497 km (14 giorni) 

                \paragraph{Mare dei Morti} (dopo 5 giorni di cammino) 
\textbf{Distanza per superare}: 4 giorni di cammino

                \paragraph{	Descrizione } Dopo 5 giorni di cammino, iniziate a notare che il terreno inizia a diventare sempre umido, l'aria inizia a diventare più pesante e sentite sempre più bzz-bzz che tanto da fastidio nelle notti estive. All'orizzonte verso sud-ovest intravedete una lunga distesa d'acqua, e a sud- est una lunga distesa di colline. Man mano che vi avvicinate all'acqua, osservate che è sporca e putrida, e in lontananza, fra la nebbia che cala sull'acqua, notate alcuni relitti e isolette con alberi spogli in cima. Odorando  sentite puzza di stantio, ma la vostra strada verso waterdeep è li difronte a voi, solo un po' più umida e tetra a causa degli alberi. 

                \paragraph{Incontri}

                        \subparagraph{Mephit del Fango}: Ogni tanto durante il loro percorso gli avventurieri scorgono al bordo della palude i mephit che si lamentano in continuazione in Aquan e Terran:  "Tesorooo, datemi i vostri tesori" "Sono cosi viscido e inutilee" "Ho tanta fameee".

                        \subparagraph{Coccodrillo}: Se provano ad immergersi, vengono attaccati da 1d4 di coccodrilli che potrebbero vedere con Percezione (16).

                        \subparagraph{Ragno Gigante}: Se nei pressi della curva decidono di entrare nella parte sinistra delle palude di 5-10 metri, fanno un tiro su percezione (12) e si accorgono nel ragno altrimenti si attaccano alla ragnatela (Forza 12 per staccarsi) e vengono attaccati da un 1 Ragno gigante. 

                        \subparagraph{Rudere e Vecchietto-Cannibale}: Verso la fine della palude c'è un vecchio rudere sulla parte sinistra del percorso, poco inoltrato e abbastanza visibile dalla strada. Mentre sono in carrozza vedono un vecchietto che sta arando il suo piccolo orticello, e mentre si avvicinano alza il capo e si avvicina alla strada e saluta con sorriso, 
cerca di convincerli a fermarsi per una tazza di te e per due chiacchiere perché vuole compagnia. 
 Ha sui 75 anni, si presenta come Jack. Jack è un vecchietto, di corporatura media, stempiato in testa ma con dei capelli grigi che formano un codino dietro la testa, 
 due occhiali che scendono sul naso. Sembra una persona dai modi gentili, molto affabile ed invita i personaggi ad entrare e a prendere una tazza di te
  (Indagare CD 17 per capire se è avvellenata, una di queste è avvelenata con Torpore (Incapacitato per 18 ore) ). Jack tenta di tenere il più possibile i pg in casa e di non farli andare via,
   dicendo che li vuole suoi ospiti per mangiare tutti insieme e che se vogliono possono dormire li in casa, non ha molte brandine ma sicuramente un posto al caldo è meglio di uno fuori nella palude.
La casa di Jack, non è enorme, ma neanche piccolissima, dentro è si vecchia ma bene arredata, 
c'è un piccolo camino con dentro un pentolone da cui proviene un buon profumo, degli scaffali con spezie e cassetti con posate e coltelli,
 poi ci sono varie statuette in legno ovunque che rappresentano corpi di uomo e donna, sotto ogni statuetta c'è un nome . 
 Se camminano per la casa , sotto un tappeto c'è una botola (Percezione 12), se chiedono cosa ci sia li sotto, si incupisce e dice "Non sono affari tuoi " in maniera rude, poi torno allegro " o meglio, nulla di interessante solo una piccolo ripostiglio con cianfrusaglie", dopo una mezz'ora il fuoco inizia a spegnersi e Jack si dilegua "Oh il fuoco si spegne, vado a fare altra legna", se uno di loro decide di accompagnarlo, accetta ben volentieri. Se fa domande insistenti Jack svia il discorso, e mentre stanno raccogliendo la legna Jack scompare, (a seconda del tiro di furtivita che fai potrebbe sapere o non sapere dove sei decidi tu se attaccare o meno). 
 Se si intrufolano nella botola, appena aprono sentono un tanfo incredibile e notano che sul coperchio della botola c'è del materiale resinoso che sembra serva per sigillare. Nella botola è buio, se non hanno scurovisione senza luce non vedono nulla, appena entrano vedono corpi appesi a dissanguarsi, un tavolo posto al centro con catene, attrezzi da macellaio appesi al muro e dei pezzi di carne sul bancone.
  Se entrano tutti, Jack li sorprende se nessuno ha perc. passiva di almeno 15, e li chiude dentro. Se invece resta qualcuno di guardia, se sono davanti la porta vedono Jack arrivare .Se Jack è solo e vede che nella casa ci sono meno persone di prima allora scappa, altrimenti si incupisce e rallenta il passo cercando di trovare una scusa. Scappa nella foresta per poi seguirli nella foresta una volta andati via.   
   


                        \subparagraph{Monti della Spada: Valle}
A valle dei monti della spada, in mezzo ai boschi che li circondano c'è un piccolo gruppo di Gnoll, se i personaggi fanno troppo rumore attirano la loro attenzione, e durante la notte verranno attaccati da 1d6 di Gnoll. 
    	   
                        
           
                                \subsubsection{Arrivo a Waterdeep : Capitolo 1}
                \paragraph{Casino in città:} Appena arrivati in città inziano subito i casi, Penelope avendo assordato un nobile è stata multata di 40 mo, mentre Al rischia di essere accusato di schiavismo verso il povero Vik, e le guardie decidono di arrestarlo e portarlo al cospetto della Corte dei Lord per essere giudicato. Lo arrestano e lo portano a Palazzo di Piergeiron, in attesa di essere giudicato lo mettono in cella. La mattina seguente viene svegliato bruscamente e portato davanti alla corte dei Lord ( ci sono tutti e 16 i lord con una maschera nera e un cappello nero con decorazioni dorate e una tunica nera)
                                        e viene accusato di schiavismo (CD 12 su persuasione per convincere i lord)
                \paragraph{Rissa in Teverna: Portale Spalancato}

                        \subparagraph{Descrizione}Siete seduti attorno a un solido tavolo di legno illuminato
dalla luce intensa di una candela e coperto di piatti
ormai svuotati e di boccali mezzi pieni. Il frastuono
dei giocatori d'azzardo che gridano e degli avventurieri
ubriachi che intonano canzoni sconce rischia di soffocare
la musica stonata di un giovane bardo che strimpella con
il suo liuto qualche tavolo più in là.
Poi un grido sovrasta tutti gli altri rumori: “Scrofa! Ti
piace uccidere i miei amici, eh?” Poi una mezzorca alta più
di due metri viene colpita da un violento pugno sferrato
da un umano la cui testa rasata è ricoperta di tatuaggi
a forma di occhio. Altri quattro umani si schierano alle
sue spalle, pronti a unirsi alla rissa. La mezzorca fa
scrocchiare le nocche delle mani, ruggisce e si avventa
sulla figura tatuata... ma prima di vedere se qualcuno
versa del sangue, un gruppo di avventori si ammassa
attorno ai lottatori e vi blocca la vista. Cosa fate?


                        \subparagraph{Rissa}I combattenti umani sono cinque membri della Gilda
di Xanathar (CM \hyperlink{banditi}{banditi} umani). Quello con gli occhi
tatuati sul cranio pelato è il loro capo, \hyperlink{krent}{ Krentz}. La loro
avversaria, \hyperlink{yagra}{Yagra Strongfist}, è una mezzorca al servizio Se
degli Zhentarim . Yagra combatte per una
questione d'orgoglio.

\begin{itemize}
    \item \textbf{Immischiarsi} Se i personaggi scelgono di unirsi alla mischia, . A Krentz rimangono soltanto 3 punti ferita
e cerca di sottrarsi alla presa di Yagra, ma gli altri quattro
membri della Gilda di Xanathar sono pronti ad avventarsi
su di lei.
Per staccare Yagra da Krentz è richiesta una prova
di Forza contesa con la prova di Forza di Yagra. Se i
personaggi la aiutano, Yagra li ringrazia, ma resta delusa
dal fatto che si siano immischiati nel combattimento.
Il DM dovrà ricordare in che modo i personaggi
interagiscono con Krentz in questa scena. Se l’uomo
sopravvive, i personaggi potrebbero incontrarlo di nuovo
in uno dei nascondigli fognari della Gilda di Xanathar (vedi
l’area Q5 a pagina 28).
\item \textbf{Tenersi Da Parte} Se i personaggi non interferiscono nella lotta, Yagra
tramortisce Krentz, ma poi i compagni dell’uomo la
malmenano fino a farla svenire. Durnan, il proprietario del
Portale Spalancato, indica loro la porta. “Fuori!” intima e i
membri della Gilda di Xanathar fuggono portandosi dietro
il corpo privo di sensi di Krentz.

\end{itemize}

                \paragraph{UN TROLL E I SUOI AMICI}
Nel terzo round della rissa, entra in scena un nuovo
problema dal pozzo aperto che si trova al centro della sala
comune del Portale Spalancato:


                        \subparagraph{Descrizione}Alcune grida d'allarme annunciano l'improvvisa
apparizione di una creatura massiccia che emerge
dal condotto al centro della sala comune: un mostro
dalla pelle verde e bozzolosa, i capelli neri lunghi e
aggrovigliati, il naso adunco lungo quanto una carota
e occhi iniettati di sangue. Quando mostra le zanne
ingiallite e ulula, notate che una mezza dozzina di
creature simili a pipistrelli sono avvinghiate al suo corpo,
mentre altre tre gli volano intorno come mosche. Tutti
gli avventori della taverna reagiscono terrorizzati, ad dell'oste, Durnan, che grida “Troll".


                        \subparagraph{Cosa succede}
Il \hyperlink{troll}{ troll}, che attualmente possiede 44 punti ferita, è salito
fin qui arrampicandosi dal primo livello di Sottomonte per
cibarsi di un po' di saporita carne umanoide, portando nove
\hyperlink{uccelli}{uccelli stigei} con sé. Una volta giunto nella sala comune,
il troll si erge in tutta la sua altezza di 2,7 metri e tira per
l'iniziativa. Anche gli uccelli stigei tirano per l'iniziativa,
ma soltanto i tre che volano sopra il troll costituiscono una
minaccia. Gli altri hanno attinto abbondantemente al sangue
del troll e sono gonfi, quindi tornano in volo all’interno del
pozzo per digerire il pasto. Man mano che il troll rigenera,
gli effetti della perdita di sangue diventano meno evidenti.
La maggior parte degli avventori e del personale della
taverna fugge o cerca riparo non appena vede il troll. Gli
uccelli stigei attaccano i personaggi più vicini, mentre
\hyperlink{Duran}{Durnan} sfodera il suo spadone,
scavalca il bancone con un balzo e affronta il mostro in
persona. Mentre attacca, ordina ai personaggi di pensare
agli uccelli stigei, poi getta addosso al troll una fiasca d'olio
per lampade a cui appicca il fuoco. Se Yagra è ancora
cosciente, si unisce allo scontro. Se qualche personaggio
contribuisce a sconfiggere il troll, Durnan lo ringrazia con
modi spicci: “Hai combattuto bene.”
Se uno o più personaggi scendono a 0 punti ferita
durante lo scontro, i membri del personale del Portale
Spalancato intervengono per stabilizzarli.

                \paragraph{Incontro con Volo} 
                \begin{itemize}
                    \item Dopo aver eliminato il troll e gli uccelli stigei, Volo si avvicina agli avventurieri.
                    \item La folla si era allontanata dal mostro e Volo li saluta lodandoli per il loro coraggio, meritato o meno.
                    \item Volo chiede se gli avventurieri sono stati mandati dal suo amico Thorgrim e afferma di avere bisogno del loro aiuto.
                    \item Propone di trovare un tavolo per discutere della sua richiesta.
                    \item Volothamp Geddarm è noto a Waterdeep come un fanfarone che abbellisce i fatti.
                    \item Nonostante ciò, Volo è una persona di buon cuore, molto legata ai suoi amici.
                    \item È preoccupato per uno di loro e, dopo un'introduzione suadente e misteriosa, la sua preoccupazione diventa evidente.
                \end{itemize}


                        \subparagraph{Descrizione} La figura che vi ha contattato si accarezza i baffi, si aggiusta il cappello spiovente e si stringe la sciarpa al collo.

                        \textit{"Volothamp Geddarm, cronachista, mago e celebrità, al vostro servizio. Immagino abbiate notato che in queste ultime decadi la città si è fatta violenta. Non vedevo tanti spargimenti di sangue dalla mia ultima visita a Baldur's Gate! Ma ora temo di avere involontariamente coinvolto un amico in questi odiosi fatti di violenza."}
                        
                        \begin{itemize}
                            \item Volo presenta il suo amico, Floon Blagmaar, descritto come più bello che intelligente.
                            \item Circa 16 giorni fa, dopo una serata a bere insieme, Floon è scomparso, probabilmente rapito.
                            \item Volo offre una ricompensa: dieci dragoni a testa ora e dieci volte tanto una volta che Floon verrà ritrovato.
                            \item Chiede agli avventurieri se può contare su di loro in questo momento di bisogno.
                        \end{itemize}
                        


\subparagraph{Cosa succede} \begin{itemize}
    \item Volo consegna a ogni personaggio un borsello contenente 10 monete d'oro (mo) per aver accettato la sua missione.
    \item Un personaggio che vuole indagare sulle sue intenzioni deve superare una prova di Saggezza (Intuizione) con CD 10.
    \item In caso di successo, il personaggio capisce che Volo è onesto, ma probabilmente esagera riguardo alla ricompensa.
    \item Volo è a corto di contanti, in attesa delle sue percentuali sugli incassi della \textit{Guida di Volo ai Mostri}, e sta lavorando a un nuovo libro: la \textit{Guida di Volo agli Spiriti e agli Spettri}, con scarsi progressi.
    \item Se messo alle strette, Volo insiste affinché i personaggi si fidino di lui e promette che il resto della ricompensa (100 mo per ciascuno) sarà pronto non appena Floon sarà riportato vivo.
\end{itemize}
\textbf{Descrizione Floon} 
\begin{itemize}
    \item Volo descrive Floon come un umano di bell'aspetto, di poco più di trent'anni, con lunghi capelli biondo-rossastri.
    \item Floon era vestito in modo elegante l'ultima volta che Volo l'ha visto.
    \item Prima della sua scomparsa, Floon e Volo stavano bevendo e facendo baldoria al \textbf{Drago Infilzato}, una taverna malfamata nel Quartiere del Porto.
    \item Volo suggerisce ai personaggi di iniziare le ricerche da quella taverna.
    \item Volo è imbarazzato nel riconoscere che potrebbe aver messo l'amico nei guai, ed esita a raccontare i dettagli di quella notte.
    \item In crisi creativa, Volo ha incontrato Floon per bere due sere prima della scomparsa. Hanno bevuto e giocato d'azzardo per qualche ora, poi Volo è andato via, e quella è stata l'ultima volta che ha visto Floon.
\end{itemize}



                        \subparagraph{Cosa E Successo QUELLA NOTTE?} .
                        \begin{itemize}
                            \item All’insaputa di Volo, dopo la sua partenza, Floon ubriaco ha incontrato un altro conoscente, Lord Renaer Neverember.
                            \item I due se ne sono andati insieme, e Renaer si è offerto di accompagnare Floon a casa.
                            \item Cinque malviventi Zhentarim al servizio di \hyperlink{urstul}{Urstul Floxin} hanno assalito sia Floon che Renaer, portandoli in un magazzino del Quartiere del Porto.
                            \item Gli Zhentarim volevano interrogare Renaer, figlio di Lord Dagult Neverember, per scoprire l’ubicazione della Pietra di Golorr e la cripta dove suo padre aveva nascosto i dragoni.
                            \item Prima dell'inizio dell'interrogatorio, alcuni membri della Gilda di Xanathar hanno teso un'imboscata agli Zhentarim, uccidendoli.
                            \item I nuovi arrivati hanno scambiato Floon per Renaer, lo hanno tramortito e portato via, mentre Renaer si è nascosto.
                            \item Floon è stato portato in un nascondiglio della Gilda di Xanathar nelle fognature.
                            \item Una banda di kenku è rimasta nel magazzino degli Zhentarim per eliminare altri Zhent che potessero arrivare.
                            \item La presenza dei kenku ha impedito a Renaer di lasciare il magazzino.
                        \end{itemize}
                        
               
\paragraph{Trovare Floon}
\begin{itemize}
    \item Volo ha visto Floon per l'ultima volta al Drago Infilzato, una taverna dubbia di proprietà degli Zhentarim, situata tra Via della Rete e Stradina del Nastro nel Quartiere del Porto.
    \item Gli incontri seguenti danno il via all'indagine dei personaggi.
    \item Sulla strada per la taverna, l'incontro "Strade Insanguinate" permette ai personaggi di vedere la Vigilanza Cittadina in azione.
    \item Se i personaggi decidono di esplorare, l'incontro "Perlustrare il Quartiere del Porto" fornisce loro un'idea dell'ambiente circostante e potrebbe condurli a scoprire "Il Negozio del Vecchio Xoblob", che richiederà un'ulteriore esplorazione.
    \item Una volta giunti al Drago Infilzato, i personaggi possono ottenere informazioni dagli avventori, che li porteranno a "Stradina delle Candele", dove la missione continua.
\end{itemize}
                        \subparagraph{Strade insanguinate}
                        \begin{itemize}
                            \item I personaggi si aggirano per il Quartiere del Porto.
                            \item Si imbattono nel teatro di uno scontro sanguinario appena concluso.
                            \item Lo scontro ha avuto luogo tra la Gilda di Xanathar e gli Zhentarim.
                        \end{itemize}

                        \textit{Quando svoltate un angolo, vi ritrovate in una via che è stata transennata dalla Vigilanza Cittadina. Sul selciato giacciono sei cadaveri, apparentemente vittime di una violenta zuffa. Gli agenti della Vigilanza hanno disarmato e arrestato tre umani fradici di sangue e stanno interrogando vari testimoni. Uno degli agenti vi nota. 
                                “Circolate” vi intima. “Qui non c'è niente da vedere.”}
\newline
\begin{itemize}
    \item La schermaglia conclusa in quest'area non è legata alla scomparsa di Floon.
    \item È invece indicativa del conflitto tra gli Zhentarim e la Gilda di Xanathar.
    \item Una dozzina di guardie della Vigilanza Cittadina ha arrestato tre banditi e sta interrogando i testimoni.
    \item Le guardie stanno aspettando i carri che porteranno via i criminali e i cadaveri.
    \item I sopravvissuti della schermaglia sono stati disarmati e fatti inginocchiare con le mani sopra la testa.
    \item I tre prigionieri sono agenti degli Zhentarim al servizio di Urstul Floxin (vedi appendice B).
    \item Saranno probabilmente accusati di omicidio.
    \item I prigionieri lanciano occhiate gelide a chiunque passi, ma la Vigilanza Cittadina impedisce ai personaggi di avvicinarsi a loro.
\end{itemize}

                                          \subparagraph{PERLUSTRARE IL QUARTIERE DEL PORTO}Il Quartiere del Porto non è un luogo sicuro. Il DM può creare l'atmosfera giusta leggendo ai giocatori il brano seguente:\newline
                                                \textit{Lunghe file di caseggiati alti e compatti lasciano buona parte dell'isolato immerso nelle ombre a pianterreno. 
                                                Quasi tutti i lampioni hanno i vetri rotti e le candele sono state rubate. L'odore di salsedine si alterna a quello degli escrementi man mano che vi inoltrate tra le lunghe file di edifici fatiscenti}\newline
                                            All’angolo tra Via Zastrow e Stradina del Nastro si trova un negozio dalla vetrina molto particolare:\newline
                                                \textit{Un negozio nei paraggi si distingue da tutti gli altri. Ha una facciata color viola scuro e nella vetrina è esibito un beholder impagliato. Sopra la porta è appesa un'insegna le cui lettere elaborate formano la scritta “Il Negozio del Vecchio Xoblob”.}\newline
                                                Se i personaggi controllano il negozio, il DM prosegue con “Il Negozio del Vecchio Xoblob”. Se non lo fanno, trovano la taverna senza altri imprevisti; vedi “Il Drago Infilzato”, più avanti.


                    \subparagraph{L NEGOZIO DEL VECCHIO XOBLOB} Quando i personaggi entrano, hanno subito l'impressione che questo sia un negozio molto strano:\newline
                        \textit{Una nube di fumo violaceo dal profumo di lavanda si diffonde fuori dalla porta del negozio quando aprite per dare un'occhiata all'interno. Ogni parete è dipinta di viola e ogni gingillo impolverato sugli scaffali è a sua volta tinto di una tonalità viola scuro. Un vecchio gnomo pelato sta seduto a gambe incrociate sul bancone e anche lui indossa abiti color prugna. Le guance sono decorate con nove occhi dipinti di color viola. 
                        Lo gnomo appoggia la pipa ed emette uno sbuffo di fumo alla lavanda prima di fare un cenno con la mano. 
                       "}\newline
                       \begin{itemize}
                        \item Il negozio è chiamato \textit{Negozio del Vecchio Xoblob} e ha un beholder impagliato in vetrina, che è in realtà un sensore magico usato da Xanathar per osservare.
                        \item Il proprietario è uno gnomo delle profondità che spia per la Gilda di Xanathar. Dopo aver sopravvissuto a un'esplosione di spore a Sottomonte, ha ereditato i ricordi di un beholder.
                        \item Desiderando un proprio dominio, lo gnomo si trasferì a Waterdeep, acquistò il negozio e cercò di rinominarlo, ma il nome originario rimase in uso.
                        \item Infine, lo gnomo cambiò il proprio nome in Xoblob e mantenne il nome del negozio come \textit{Negozio del Vecchio Xoblob}.
                    \end{itemize}
                        \begin{itemize}
                        

                            \item \textbf{Oggetti Insoliti} 
                        \begin{itemize}
                                \item Lampada gnomesca con canarino meccanico (che se accessa suona faccetta nera) 6 mo 
                                \item Una pergamena su cui è stato disegnato un complicato
                                marchingegno (vibratore a energia magica) 3 mo
                                \item Una piramide di bastoncini di incenso nero dal pessimo
                                odore 3 mo
                                \item Un distintivo d'argento a forma dl una stella a cinque
                                punte 2 mo
                                \item Un occhio di vetro 2 mo
                            \end{itemize}
                        \item \textbf{Il Destino di Floon.}
                        \begin{itemize}
                            \item Lo gnomo non conosce Floon per nome, ma lo riconosce dalla descrizione.
                            \item Esita a rivelare ciò che sa.
                            \item Se qualcuno gli offre un nuovo oggetto viola o effettua con successo una prova di Carisma (Intimidire o Persuasione) con CD 13, può convincerlo a parlare.
                            \item Lo gnomo racconta che Floon e un altro uomo ben vestito (Renaer Neverember, anche se non lo ha riconosciuto) sono stati assaliti fuori dal negozio.
                            \item Gli aggressori erano uomini minacciosi con armature di cuoio nere.
                            \item Lo gnomo crede di aver visto cinque aggressori, ma nessuno di loro gli è parso familiare.
                            \item Uno degli aggressori aveva un tatuaggio nero di un serpente alato sul collo.
                        \end{itemize}
                        
                        \end{itemize}


                        \subparagraph{L DRAGO INFILZATO}
Il Drago Infilzato si affaccia su un vicolo che collega Via della Rete e Stradina del Nastro nel Quartiere del Porto, a poca distanza dal Negozio del Vecchio Xoblob. Quando i personaggi si avvicinano alla taverna, il DM legge il brano seguente:\newline
\textit{Il Drago Infilzato sembra un edificio in rovina. Entrambe le finestre che danno sulla strada sono sfondate e l'ancora di una nave è incastrata nel tetto. Oltre le finestre, notate un gruppo di avventori macilenti che bevono da enormi boccali.}\newline
Floon non è stato visto al Drago Infilzato fin dalla notte della sua sparizione e gli scaricatori di porto che frequentano il locale non amano parlare con gli sconosciuti. Per farli spottonare, sarà necessario offrire loro qualche soldo o effettuare con successo una prova di Carisma (Intimidire o Persuasione) con CD 13.\newline
\textbf{Il Destino di Floon.} \begin{itemize}
    \item Alcuni avventori abituali ricordano di aver visto Volo e Floon bere insieme un paio di settimane fa.
    \item Dopo che Volo è andato via, Floon è rimasto nel locale per incontrarsi con Renaer Neverember, figlio del precedente Lord Svelato di Waterdeep, Dagult Neverember.
    \item Gli avventori descrivono Renaer come un nobile ricco e viziato, simile a suo padre.
    \item Floon e Renaer hanno bevuto e giocato a Tre Draghi al Buio prima di andarsene attorno a mezzanotte.
    \item Cinque uomini li hanno seguiti, e nessuno nella taverna sa cosa sia accaduto poi.
    \item Gli uomini che sono usciti subito dopo Floon e Renaer non sono più tornati alla taverna.
    \item Si dice che frequentino un magazzino nella Stradina delle Candele.
    \item Un avventore suggerisce di cercare la porta con il simbolo del serpente.
\end{itemize}

                        \subparagraph{STRADINA DELLE CANDELE}
Gli edifici su entrambi i lati di Stradina delle Candele sono così alti e così vicini gli uni agli altri che la luce sfiora il fondo della via soltanto a solealto.\newline
\textit{Questo stretto vicolo è talmente buio che sembra quasi il tunnel di un dungeon... ed emana anche la stessa puzza. 
Quasi tutti i lampioni sono stati spaccati. L'unica luce a brillare nell'oscurità è una debole fiamma tremolante in fondo al vicolo, simile a una lontana candela.}
\newline
La luce tremolante proviene dall'unico lampione ancora intatto di Stradina delle Candele, tenuto acceso da un incantesimo fiamma perenne. Sul lato della strada di fronte alla lampada è situato un magazzino. Sulla porta, illuminata dal lampione, è visibile l'immagine di un serpente nero alato (il simbolo degli Zhentarim) disegnato appena sopra la maniglia. Se un personaggio ha un legame con gli Zhentarim, riconosce subito il simbolo, mentre gli altri possono ricordarne il significato, se effettuano con successo una prova di Intelligenza con CD 10. 
\textbf{Simbolo Zhentarim: }Il simbolo degli Zhentarim raffigura un serpente nero con ali spiegate. Il serpente è sinuoso, con il corpo arrotolato o a forma di "S", mentre le ali, simili a quelle di un drago, sono estese lateralmente. Il colore dominante è il nero, che ricopre sia il corpo del serpente che le ali, creando un'immagine minacciosa e oscura. Il simbolo è spesso presentato su uno sfondo neutro o scuro, evidenziando i dettagli intricati del serpente e delle ali.
                \paragraph{Il nascondiglio degli Zhentarim}
IIl nascondiglio di Stradina delle Candele (mappa 1.1) è un fatiscente magazzino a due piani. La Rete Nera ha altri rifugi simili in tutta Waterdeep, per cui questa planimetria può essere riutilizzata per altri nascondigli degli Zhent.

\begin{itemize}
    \item Il magazzino si trova in un cortile esterno circondato da un’alta recinzione, con un cancello non chiuso a chiave.
    \item Ci sono tre punti d'accesso: una porta principale, una porta per lo scarico delle merci e una finestra riverniciata, tutti chiusi a chiave.
    \item La porta principale ha uno spioncino apribile dall'interno. Le serrature delle porte e della finestra possono essere aperte con una prova di Destrezza (CD 12) usando arnesi da scasso, o forzate con una prova di Forza (Atletica) (CD 10).
    \item Bussare mette in allarme i kenku all'interno, che si nascondono dietro i mobili. Un personaggio con Saggezza (Percezione) passiva pari o superiore a 16 può sentire il loro movimento.
    \item I kenku appartengono alla Gilda di Xanathar, che ha ucciso quasi tutti gli occupanti del magazzino. Dopo che cinque Zhentarim hanno catturato Renaer Neverember e Floon Blagmaar, Floon è stato portato via, mentre Renaer si è salvato nascondendosi.
    \item Renaer sta cercando di sfuggire ai kenku, che perlustrano il magazzino svogliatamente in attesa di altri Zhent.
\end{itemize}

                        \subparagraph{Z1. CAMERA PRINCIPALE}
                        La Rete Nera si occupa principalmente di reclutare, addestrare ed equipaggiare mercenari. Il magazzino è pieno di casse con armi, razioni, stivali, uniformi nere ed equipaggiamento vario.

                        \begin{itemize}
                            \item Quando i personaggi cercano di entrare, il DM deve determinare se i quattro \hyperlink{kenk}{kenku} all'interno sono consapevoli della loro presenza prima di leggere il testo del riquadro.
                            \item Un personaggio con arnesi da scasso può scassinare una serratura su una porta o finestra con una prova di Destrezza (CD 10).
                            \item Se i personaggi riescono a entrare di soppiatto, possono cercare di sorprendere i kenku. Se bussano o annunciano il loro arrivo, i kenku si nascondono come descritto in precedenza.
                        \end{itemize}

\textit{| tavoli e le sedie sono stati rovesciati e sparpagliati a
terra. Lungo le pareti giacciono i cadaveri di una dozzina
di uomini, i cui stocchi e pugnali sono rimasti a terra
accanto a loro. Sul lato nord dell'area, una fila di scale
sale fino al livello aperto superiore.} \\
Se i kenku non sono nascosti, il DM aggiunge:
\textit{Quattro basse creature simili a uccelli, dal lungo becco
e dalle piume nere, alzano lo sguardo colti di sorpresa
nel punto in cui si trovavano, al centro del magazzino.
Ognuno indossa un mantello dotato di cappuccio e
brandisce una spada corta.}\\

I cadaveri appartengono a cinque mercenari umani degli Zhentarim (gli stessi che hanno rapito Floon e Renaer) e sette malviventi umani della Gilda di Xanathar. Tutti indossano armature di cuoio.

\begin{itemize}
    \item Gli Zhentarim hanno un tatuaggio nero di un serpente alato sul collo o sull'avambraccio.
    \item Un membro della Gilda di Xanathar ha un tatuaggio nero sul palmo della mano destra che rappresenta un cerchio con dieci raggi (il simbolo di Xanathar).
    \item I kenku combattono finché due di loro non sono incapacitati o uccisi. I sopravvissuti tenteranno di fuggire.
    \item Una prova di Carisma (Intimidire) con CD 10 costringerà i kenku catturati a rivelare ciò che sanno.
\end{itemize} 

\textbf{Cosa SANNO I KENKU} Quando i kenku parlano, imitano i suoni e le voci che
hanno sentito in precedenza. Se interrogati, ripetono le
frasi seguenti: 
\begin{itemize}
    \item Con una voce profonda dal marcato accento orchesco:
    “Xanathar manda i suoi saluti.”
    \item Con una voce acuta e nasale: “Legate il bel giovanotto
    nella stanza sul retro!" e “Seguite i segnali gialli
    nelle fognature.” (Questa osservazione si riferisce ai
    tunnel nelle fognature contrassegnati con il simbolo di
    Xanathar, che conducono al nascondiglio della Gilda di
    Xanathar.)
    \item Con voce rauca: “Non c'è tempo per saccheggiare il
    posto. Portatelo dal capo e basta.”

\end{itemize}


                        \subparagraph{Z2. RIPOSTIGLIO}
                        La porta di questa stanza sul retro pende precariamente su un paio di cardini sfondati. Dalla stanza oltre la porta proviene un intenso fetore di pesce andato a male e di aceto. 

                        \begin{itemize}
                            \item La stanza è piena di corde di scarto, teli cerati e assi di legno spezzate, ricavate da barili danneggiati.
                            \item Renaer Neverember (vedi l'appendice B) si è nascosto qui dopo essersi liberato dai legacci.
                            \item I personaggi possono sentire il suo respiro ansimante provenire da sotto un telo all'estremità nord della stanza.
                        \end{itemize}

\textbf{Intepretare Renar}:\hyperlink{Renaer}{Renaer} è disarmato, sporco di fango e di fetide macchie di aringhe sott'olio andate a male. Parla in modo aggraziato ed eloquente, come si addice a un nobile della sua levatura. Concede facilmente la sua fiducia, ma se viene tradito, non la concederà mai una seconda volta.

\begin{itemize}
    \item La notte del rapimento, Renaer temeva che Floon fosse troppo ubriaco per tornare a casa da solo e si è offerto di accompagnarlo. I due sono stati aggrediti da cinque malviventi subito dopo aver lasciato Stradina del Nastro e essersi diretti a nord in Via Zastrow.
    \item Renaer si sente in colpa per il rapimento di Floon, credendo (a buon motivo) che i rapitori abbiano scambiato Floon per lui.
    \item Se i personaggi chiedono a Renaer di unirsi alle ricerche di Floon, il giovane nobile acconsente e si procura un pugnale e uno stocco tra quelli appartenenti agli Zhent uccisi nel magazzino.
    \item Se un personaggio chiede a Renaer perché gli Zhent lo abbiano rapito, il giovane risponde sinceramente con queste parole:
\end{itemize}


\textit{“Gli Zhentarim pensano che mio padre abbia sottratto
una grossa quantità d'oro quando era il Lord Svelato
e che abbia nascosto i dragoni da qualche parte qui
in città. Pensano che sia possibile trovarli usando un
artefatto chiamato la Pietra di Golorr, che fino a poco
tempo fa era nelle mani della Gilda di Xanathar. A quanto
pare, qualcuno l'ha rubata. Gli Zhent pensavano che io
sapessi qualcosa di tutto questo, ma non ne so niente. lo
e mio padre non ci parliamo da anni.”}


                        \subparagraph{Z3. STANZA SEGRETA}
                        Questa stanza è nascosta dietro una porta segreta che può essere trovata con una prova di Saggezza (Percezione) con CD 15. Quando la porta segreta viene aperta, i personaggi sentono il lontano rumore di un campanello che suona negli uffici soprastanti (area Z5).

                        \textbf{Tesoro}: Gli Zhent hanno nascosto in questa stanza due casse di legno.
                        
                        \begin{itemize}
                            \item La prima cassa, rubata dai moli, contiene quattro dipinti incorniciati in legno e avvolti in una tela di pelle. I dipinti raffigurano le città di Luskan, Neverwinter, Silverymoon e Baldur's Gate e valgono 75 mo ciascuno.
                            \item La seconda cassa, rubata da una carovana sulla Strada Alta, contiene quindici lingotti d'argento da 5 kg l'uno. Sebbene siano anneriti e corrosi, valgono ancora 50 mo ciascuno.
                        \end{itemize}
                        


                        \subparagraph{Z4. BALCONE}
                        La parte aperta del piano superiore si affaccia sul magazzino principale ed è anch'essa piena di casse. Se i personaggi cercano tra le casse, trovano ogni genere di cianfrusaglie, tra cui:

                        \begin{itemize}
                            \item Rotoli di stoffa divorati dalle tarme,
                            \item Bottiglie di olio d'oliva andato a male,
                            \item Centinaia di paia di sandali con suola di legno, che l'estate scorsa erano l'ultimo grido della moda ma ora sono snobbati.
                        \end{itemize}
                        
                        Tutta questa merce è priva di valore.
                        


                        \subparagraph{Z5. UFFICI}
Il piano superiore ospita uffici poco utilizzati dagli Zhent, contenenti scrivanie, sedie e scaffali vuoti, ricoperti di polvere e ragnatele. Qualche topo innocuo si aggira tra le stanze. 

Sopra la porta di ogni ufficio è montato un campanello d'allarme in acciaio, collegato tramite cavi alla porta segreta dell'area Z3. I campanelli suonano rumorosamente quando quella porta viene aperta.

\textbf{TESORO}
Setacciando gli uffici, un personaggio trova un uccello di carta inutilizzato

\textbf{Uccello di carta}
Dopo che il personaggio ha scritto un messaggio di cinquanta parole o meno su questo foglio magico di carta pergamenata e ha pronunciato il nome di una creatura, il foglio si piega magicamente per diventare un uccello di carta Minuscolo e vola fino al destinatario indicato dal personaggio. Il destinatario deve trovarsi sullo stesso piano di esistenza del personaggio; altrimenti, l'uccello si trasforma in cenere al momento di spiccare il volo.

\textbf{Statistiche dell'uccello di carta:}
\begin{itemize}
    \item Punti ferita: 1
    \item Classe Armatura (CA): 13
    \item Velocità di volo: 18 metri
    \item Destrezza: 16 (+3)
    \item Punteggi in altre caratteristiche: 1 (-5)
    \item Immunità: Danni psichici e da veleno
\end{itemize}
                \paragraph{ARRIVA LA VIGILANZA}
                \begin{itemize}
                    \item Dopo il salvataggio di Renaer, il capitano Hyustus Staget (LB veterano umano Illuskan) e una dozzina di veterani irrompono nel magazzino dopo una segnalazione di attività sospette.
                    \item Se ci sono ancora kenku vivi, vengono messi sotto custodia. I morti sono identificati come membri degli Zhentarim e della Gilda di Xanathar.
                    \item Il Capitano Staget, noto e rispettato nel Quartiere del Porto, mantiene un comportamento impeccabile a causa del coinvolgimento di Renaer.
                    \item Staget non interferisce direttamente nei conflitti tra Gilda di Xanathar e Zhentarim, ma cerca di mantenere la pace.
                    \item Un tempo sorvegliava il magazzino, ma ha richiamato le sentinelle per rinforzare le pattuglie nel Quartiere del Porto, ora rimpiangendo la decisione.
                    \item Staget e Renaer si riconoscono di vista. Il capitano consegna ai personaggi un foglio con il Codice Legale, incoraggiandoli a leggerlo.
                    \item Se i personaggi chiedono aiuto per rintracciare Floon, Staget chiarisce che non invierà i suoi uomini nelle fognature per cercare spie degli Zhent o della Gilda di Xanathar.
                    \item Se i personaggi si immischiano ulteriormente nel conflitto, Staget offre qualche consiglio amichevole prima di lasciarli andare.
                \end{itemize}
\begin{itemize}
    \item “Meglio non immischiarsi nelle questioni dei criminali.
    Lasciate questo lavoro sporco alla Vigilanza Cittadina.”
    \item “Non tutti gli agenti della Vigilanza Cittadina sono
    indulgenti quanto me.”
    \item “Tenete il sangue lontano dalle strade, chiaro?” (Questo è
    un detto comune tra gli ufficiali della Vigilanza Cittadina,
    che si preoccupano prima di tutto di ciò che accade in
    superficie e non si curano di cosa capita nelle fognature
    sotterranee).
\end{itemize}

Se i personaggi provocano altri guai nel Quartiere del
Porto probabilmente si imbatteranno di nuovo nel Capitano
Staget. Anche se sotto sotto il capitano non disdegna il fatto
che degli avventurieri sbrighino parte del lavoro per lui,
non può consentire che mettano in ombra le sue iniziative
per mantenere la pace senza rischiare una nota di biasimo
dai suoi superiori.
 
\end{document}
