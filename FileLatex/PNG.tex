\documentclass{article}
\usepackage{graphicx}
\usepackage{hyperref}
\usepackage{imakeidx}
\usepackage{rotating}
\usepackage{multirow}
\usepackage{array}
\usepackage[export]{adjustbox}
\usepackage{geometry}
\begin{document}

\paragraph{Nome\_PNG}Xoblob
\paragraph{Razza} gnomo delle profondità
\paragraph{Classe} -
\paragraph{Descrizione}  Il negoziante è un vetusto gnomo delle profondità che spia per la gilda di Xanathar. Alcuni anni fa, sopravvisse alla detonazione di una nube di spore a Sottomonte ed ereditò i ricordi di un beholder di passaggio. Spinto da un ossessivo desiderio di crearsi un suo dominio, lo gnomo si stabilì a Waterdeep, comprò il Negozio del Vecchio Xoblob dal proprietario precedente e cercò di ribattezzarlo con il suo nome, ma tutti continuarono a chiamarlo il Negozio del Vecchio Xoblob. Allora lo gnomo restituì al negozio il vecchio nome e cambiò il proprio in Xoblob.
\paragraph{Cosa fa } 
    \subparagraph{Cosa ha da vendere}
    \begin{itemize}
        \item \textbf{Oggetti Insoliti} \begin{itemize}
                \item Lampada gnomesca con canarino meccanico (che se accessa suona faccetta nera) 6 mo 
                \item Una pergamena su cui è stato disegnato un complicato
                marchingegno (vibratore a energia magica) 3 mo
                \item Una piramide di bastoncini di incenso nero dal pessimo
                odore 3 mo
                \item Un distintivo d'argento a forma dl una stella a cinque
                punte 2 mo
                \item Un occhio di vetro 2 mo
            \end{itemize}
        \end{itemize}
    \subparagraph{Cosa sa di FLoon}
    
                        \begin{itemize}
                            \item Lo gnomo non conosce Floon per nome, ma lo riconosce dalla descrizione.
                            \item Esita a rivelare ciò che sa.
                            \item Se qualcuno gli offre un nuovo oggetto viola o effettua con successo una prova di Carisma (Intimidire o Persuasione) con CD 13, può convincerlo a parlare.
                            \item Lo gnomo racconta che Floon e un altro uomo ben vestito (Renaer Neverember, anche se non lo ha riconosciuto) sono stati assaliti fuori dal negozio.
                            \item Gli aggressori erano uomini minacciosi con armature di cuoio nere.
                            \item Lo gnomo crede di aver visto cinque aggressori, ma nessuno di loro gli è parso familiare.
                            \item Uno degli aggressori aveva un tatuaggio nero di un serpente alato sul collo.
                        \end{itemize}
   
\paragraph{Cosa sa/ Discorsi}
    \begin{enumerate}
        \item   “Salute a voi, bentrovati! Venite a dare un'occhiata alla merce del più curioso negozio di curiosità del mondo!  “Nessuna parentela con l'occhio tiranno appeso in vetrina!” si affretta sempre ad aggiungere.
    \end{enumerate}

\paragraph{Carattere} -
\paragraph{Abilità } -
\end{document}